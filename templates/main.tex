% !TEX TS-program = XeLaTeX

% LianTze Lim templates helped a lot!
% https://www.overleaf.com/latex/examples/how-to-write-multilingual-text-with-different-scripts-in-latex/wfdxqhcyyjxz
% https://www.overleaf.com/blog/441-how-to-write-in-markdown-on-overleaf


\documentclass[a4paper]{book}%

\documentclass{article}%


\usepackage{geometry}

% info.target.reset_footer_active
% info.target.reset_footer_depth: 0

% info.target.clear_page_active: true
% info.target.clear_page_depth: 1






% shortcuts
\usepackage{xcolor}
% \def \utfbox {\nopagebreak\hfill □}
\def \utfboxraw {\hfill\color{black!20} □\color{black}}
\def \utfbox {\nopagebreak\utfboxraw}
\def \texthash {\#}

\def \endsec {\utfbox\clearpage}
\def \endfoot {\setcounter{footnote}{0}}

\setcounter{tocdepth}{->{{ info_target.toc_depth }}}






\usepackage[footnotes,hybrid,underscores=false]{markdown}
%\usepackage[hybrid]{markdown}
\markdownSetup{
	rendererPrototypes={
    	link = {(links nor wotking for now)}
    	% link = {\href{->#2}{->#1}\footnoteB{\href{->#2}{->{\ttfamily\scriptsize\relax$ \langle $#2$ \rangle $}}}}
	},
	renderers = {
		olItemWithNumber = {\item},
	}
}



\usepackage[hidelinks]{hyperref}


\usepackage[protrusion=true,final]{microtype}



% \usepackage{xepersian}
% \defpersianfont\nastaligh[Language=Default]{IranNastaliq}
% \renewcommand{\LettrineFontHook}{\nastaligh\color{red}}

% https://ctan.org/pkg/xecyr % Using Cyrillic languages in XeTeX
% https://ctan.org/pkg/xgreek % XeLaTeX package for typesetting Greek language documents (beta release)



\usepackage{lettrine}



\input Zallman.fd
\newcommand*\initialsfamily{\usefont{U}{Zallman}{xl}{n}}
\renewcommand{\LettrineFontHook}{\initialsfamily}



% prev loversize: +0.115
\newcommand{\DECORATE}[3][]{\lettrine[lines=3,loversize=-.055,#1]{\disableTransitionRules#2\enableTransitionRules}{->#3}}

% \newfont{\initial}{wcmr17 at 48pt}
% \newcommand{\frstltr}[1]{
%     \newbox{\litera}
%     \savebox{\litera}{\hbox #1}
%     % \savebox{\litera}{\hbox{\initial #1}}
%     \vspace*{.2\ht\litera}\par\noindent
%     \begin{wrapfigure}{l}{.8\wd\litera}
%     \vbox to .05\ht\litera{\vss\usebox{\litera}\vspace*{-.65\ht\litera}}
%     \vspace*{-.2\ht\litera}
%     \end{wrapfigure}}
% \frstltr{Д}олжно работать!




\begin{document}%

% \DECORATE{T}{esting} the thing. The brown fox jumps over the lazy dog. The brown fox jumps over the lazy dog. The brown fox jumps over the lazy dog. The brown fox jumps over the lazy dog. The brown fox jumps over the lazy dog. The brown fox jumps over the lazy dog. The brown fox jumps over the lazy dog. The brown fox jumps over the lazy dog. The brown fox jumps over the lazy dog. The brown fox jumps over the lazy dog. The brown fox jumps over the lazy dog. The brown fox jumps over the lazy dog. The brown fox jumps over the lazy dog. The brown fox jumps over the lazy dog. The brown fox jumps over the lazy dog. The brown fox jumps over the lazy dog. 

% \DECORATE{T}{esting} the thing. The brown fox jumps over the lazy dog. The brown fox jumps over the lazy dog. The brown fox jumps over the lazy dog. The brown fox jumps over the lazy dog. The brown fox jumps over the lazy dog. The brown fox jumps over the lazy dog. The brown fox jumps over the lazy dog. The brown fox jumps over the lazy dog. The brown fox jumps over the lazy dog. The brown fox jumps over the lazy dog. The brown fox jumps over the lazy dog. The brown fox jumps over the lazy dog. The brown fox jumps over the lazy dog. The brown fox jumps over the lazy dog. The brown fox jumps over the lazy dog. The brown fox jumps over the lazy dog. 



\tableofcontents


	\clearpage




\begin{markdown}

\end{markdown}
\markdownInput{->{{md.0}}}%
\begin{markdown}

\end{markdown}


\end{document}
