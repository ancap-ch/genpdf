% eg.: en



% eg.: th jp
% 
% {{ lang }}
% 

% % Good links
% % https://en.wikibooks.org/wiki/LaTeX/Internationalization
% % https://en.wikibooks.org/wiki/LaTeX/Fonts


    

    \TODOLangPtBR

    \TODOLangElse



% OT1 (TeX text) is likely the default, more ascii-sh like
% OT2 (U.Washington cyrillic enc)
% OT3 (U.Washington IPA enc)
% OT4 (Polish text enc)
% OT6 (Armenian text enc)
% T1 (Cork enc)
% T2A, T2B, T2C (Cyrillic encs) but see X2, which includes all
% T3 (IPA enc) for japanese, but should actually be called X3
% T4 (African Latin fc)
% T5 (Vietnamese enc)
% T6 (Armenian)
% T7 (Greek)
% TS1 (Cork) for symbols
% TS3 (IPA symbol enc)
% X2 (Cyrillic glyph container) extended
% C.. (CJK enc)
% E.. (experimental enc)
% L.. (local enc)
% LGR (Greek) by MicroPress
% PD1 (PDF DocEncoding) by Adobe
% PU (PDF Unicode enc) by Adobe

% % the babel pdf says you may use various input encodings with:
% \addto\extrasfrench{\inputencoding{latin1}}
% \addto\extrasrussian{\inputencoding{koi8-r}}

% % obersvations
%
% % T1 is good for latin
% % babel definition should be used soon after document class!
%
% % ------
% % for arabic script: 
% % Arabic, Persian, Urdu, Pashto, Kurdish, Uyghur, etc.
% \usepackage{arabtex}
% % or use Arabi package
% \usepackage{cmap} % enables copy-pasting
% \usepackage[LAE,LFE]{fontenc}
% \usepackage[arabic,farsi]{babel}
%
% % ------
% % Armenian
% % appears to work best in xelatex
%
% % ------
% % Cyrillic
% % Bulgarian, Russian and Ukrainian
% % something like:
% \usepackage[T1,T2A]{fontenc}
% \usepackage[english,bulgarian,russian,ukrainian]{babel}
% % and just load the fonts, babel will switch when appropriate
% % the old xelatex had some cyr stuff too (xecyr pack)
%
% % ------
% % Chinese
% \usepackage{CJK}
% \begin{document}
% \begin{CJK}{UTF8}{gbsn} % gbsn is the font
% 你好
% You can mix Latin letters and Chinese.
% \end{CJK}
% \end{document}
%
% % other fonts are: 
% % gbsn (简体宋体, simplified Chinese)
% % gkai (简体楷体, simplified Chinese)
% % bsmi (繁體細上海宋體, traditional Chinese)
% % bkai (繁體標楷體, traditional Chinese)
%
% % ------
% % czech
% \usepackage[czech]{babel}
% % note that quotes are „czech quotation marks“
% % if copy-pasting is necessary, see the I10n part for czech
% % it will use cmap etc packages, depending on the font
%
% % ------
% % Finnish
% \usepackage[finnish]{babel}
% 
% % ------
% % French
% % better to load before babel
% \documentclass[french]{article}
% \usepackage[utf8]{inputenc}
% \usepackage[T1]{fontenc}
% \usepackage{lmodern} % the font..
% \usepackage{babel} % but \usepackage[frenchb]{babel} is also possible
% % the textcomp pack will fix the ° char
%
% % ------
% % German
% \usepackage[german]{babel}
% % or \usepackage[ngerman]{babel} for "new" german orthography
% % btw this is a quote in german »this« (but in Switzerland its «this»)
% % also use the T1 encoding
% % ziffer pack enables to use numbers like 0,5 
% % (instead of having to use like 0{,}5 )
%
% % ------
% % Greek
% % \usepackage[english,greek]{babel} % just an example
% % \usepackage[iso-8859-7]{inputenc} % quite different
%
% % the old xelatex had greek comments too (xgreek pack)
% % maybe this also works for latex..
%
% % ------
% % Hungarian
% \usepackage[utf8]{inputenc}
% \usepackage{t1enc} % don't know if this is different from T1
% % \usepackage[latin2]{inputenc} % maybe this can work as well
% \def\magyarOptions{defaults=hu-min} % maybe this is not needed
% % \def\magyarOptions{defaults=compat-1.4} % other options..
% % \def\magyarOptions{defaults=safest}
% \usepackage[magyar]{babel}
%
% % ------
% % Icelandic and Faroese
% \usepackage[icelandic]{babel}
%
% % ------
% % Italian
% \usepackage[italian]{babel}
%
% % ------
% % Norwegian
% \usepackage[norsk]{babel}
%
% % ------
% % Japanese
% % check the tex package pTeX.
% % The latex cjk support is old, but works like the chinese one
% \documentclass{article}
% \usepackage{CJK}
% \begin{document}
% \begin{CJK}{UTF8}{min} % the min is the font
% こんにちは
% You can mix latin letters as well as hiragana, katakana and kanji.
% \end{CJK}
% \end{document}
%
% % ------
% % Korean
% \usepackage{hangul} % for HLATEX (which is different form the CJK)
% % there is also another one, hLATEXp
% % for just the korean font, one may use:
% \usepackage{hfont}
% % they also have a linux group where they recommend the pack kotex
% % check it later..
%
% % ------
% % Persian
% % there is persian in another section already.
% % but they really recommend xelatex and the pack xepersian
%
% % the old xelatex had persian comments as well..
% \defpersianfont\nastaligh[Language=Default]{IranNastaliq}
% \renewcommand{\LettrineFontHook}{\nastaligh\color{red}}
% 
% % ------
% % Polish
% \usepackage{polski}
% \usepackage[polish]{babel}
% % they also recommend using mwart over article,
% % mwbk over book,
% % mwrep over report
% % but since I'm in memoir, maybe I won't use it. (low priority)
%
% % ------
% % Portuguese
% \usepackage[portuguese]{babel}
% % \usepackage[brazil]{babel} % for the brazilian case..
%
% % ------
% % Slovak
% \usepackage[slovak]{babel}
% \usepackage[IL2]{fontenc}
%
% % ------
% % Spanish
% % \def\spanishoptions{mexico} % some options may be loaded..
% \usepackage[spanish]{babel}
%
% % ------
% % Thai
% % better use xetex with babel or polyglossia
% % for pdftex (idk what this is), an external tool should do the hyphenation
% % stuff
%
% % example for xetex, using babel:
% \documentclass{book}
% \usepackage{babel}
% \babelprovide[main, import]{thai}
% \babelfont{rm}{FreeSerif}
% \begin{document}
% ปัจจุบันข้าวและพริกเป็นส่วนประกอบสำคัญที่สุดของอาหารไทย
% \end{document}
%
% % ------
% % Tibetan
% \usepackage{ctib}
%
% % ------
% % Vietnamese
% \documentclass{article}
% \usepackage{fontspec}%
% \setmainfont[Ligatures=TeX]{Linux Libertine O}
% % appearently it doesnt indicates babel..
% % but there is viet on babel..
% % there is also the vntex pack
% 
%


% useful commands:
%
% % TODO: all languages that are to be used must be declared
% % at the beggining
% % and the last one is the main one
%
% \selectlanguage{czech}
% % changes the "complete setup" to the language "from now on"
% % for a kind of less-instrudive, you may do a:
% {\selectlanguage{czech} blablabla }\selectlanguage{english}
%
% \selectlanguage{english}
% % changes the "complete setup" to the language "from now on"
%
% % if more power is needed, such as changing left->right to right->left,
% % use something like
% \begin{otherlanguage}{hebrew}
%  blablabla
% \end{otherlanguage}
%
%
%
% % if only a small snippet is desired, without changing "too much", then
%
% \begin{otherlanguage*}{thai}
%  blablabla
% \end{otherlanguage*}
% % you may also enclose with braces, so it's more "local"
%
% % or else,
%
% \foreignlanguage{thai}{blablabla}
% % this does not change names nor dates
% % does not really require the language to be loaded
% 
